%%%
% Plantilla de Presentación
% Modificación de una plantilla de Latex de LaTeXTemplates para adaptarla 
% al castellano y a las necesidades de escribir informática y matemáticas.
%
% Editada por: Mario Román
%
% License:
% CC BY-NC-SA 3.0 (http://creativecommons.org/licenses/by-nc-sa/3.0/)
%%%

%%%%%%%%%%%%%%%%%%%%%%%%%%%%%%%%%%%%%%%%%
% Beamer Presentation
% LaTeX Template
% Version 1.0 (10/11/12)
%
% This template has been downloaded from:
% http://www.LaTeXTemplates.com
%
% License:
% CC BY-NC-SA 3.0 (http://creativecommons.org/licenses/by-nc-sa/3.0/)
%
%%%%%%%%%%%%%%%%%%%%%%%%%%%%%%%%%%%%%%%%%

%----------------------------------------------------------------------------------------
%   PAQUETES Y CONFIGURACIÓN DEL DOCUMENTO
%----------------------------------------------------------------------------------------

\documentclass{beamer}

%% Configuración de la presentación
\mode<presentation> {
  %%% Selección de estilo
  % The Beamer class comes with a number of default slide themes
  % which change the colors and layouts of slides. Below this is a list
  % of all the themes, uncomment each in turn to see what they look like.

  % \usetheme{default}
  % \usetheme{AnnArbor}
  \usetheme{Antibes}
  %\usetheme{Bergen}
  %\usetheme{Berkeley}
  %\usetheme{Berlin}
  %\usetheme{Boadilla}
  %\usetheme{CambridgeUS}
  %\usetheme{Copenhagen}
  %\usetheme{Darmstadt}
  %\usetheme{Dresden}
  %\usetheme{Frankfurt}
  %\usetheme{Goettingen}
  %\usetheme{Hannover}
  %\usetheme{Ilmenau}
  %\usetheme{JuanLesPins}
  %\usetheme{Luebeck}
  % \usetheme{Madrid}
  %\usetheme{Malmoe}
  %\usetheme{Marburg}
  %\usetheme{Montpellier}
  %\usetheme{PaloAlto}
  %\usetheme{Pittsburgh}
  %\usetheme{Rochester}
  %\usetheme{Singapore}
  %\usetheme{Szeged}
  %\usetheme{Warsaw}

  %% Selección de color
  % As well as themes, the Beamer class has a number of color themes
  % for any slide theme. Uncomment each of these in turn to see how it
  % changes the colors of your current slide theme.

  % \usecolortheme{albatross}
  % \usecolortheme{beaver}
  %\usecolortheme{beetle}
  %\usecolortheme{crane}
  \usecolortheme{dolphin}
  %\usecolortheme{dove}
  %\usecolortheme{fly}
  %\usecolortheme{lily}
  %\usecolortheme{orchid}
  %\usecolortheme{rose}
  %\usecolortheme{seagull}
  %\usecolortheme{seahorse}
  %\usecolortheme{whale}
  %\usecolortheme{wolverine}

  %% Configuración del pie de línea
  %\setbeamertemplate{footline} % To remove the footer line in all slides uncomment this line
  %\setbeamertemplate{footline}[page number] % To replace the footer line in all slides with a simple slide count uncomment this line
  %\setbeamertemplate{navigation symbols}{} % To remove the navigation symbols from the bottom of all slides uncomment this line
}

%% Fuentes de tamaño arbitrario
\usepackage{lmodern}

%% Gráficos
\usepackage{graphicx} % Allows including images
\usepackage{booktabs} % Allows the use of \toprule, \midrule and \bottomrule in tables

%%% Castellano.
% noquoting: Permite uso de comillas no españolas.
% lcroman: Permite la enumeración con numerales romanos en minúscula.
% fontenc: Usa la fuente completa para que pueda copiarse correctamente del pdf.
% \usepackage[spanish,es-noquoting,es-lcroman]{babel}
\usepackage[utf8]{inputenc}
\usepackage[T1]{fontenc}
% \selectlanguage{spanish}

%Configuración de tablas
\usepackage{longtable}
\usepackage{rotating}
\usepackage{array}
\usepackage{multicol}
\usepackage{multirow}
\usepackage{booktabs}
\usepackage{float}

%----------------------------------------------------------------------------------------
%   TÍTULO
%----------------------------------------------------------------------------------------

\title[Método de Euler para sist. de ecuaciones]{Método de Euler para sistemas de ecuaciones diferenciales} % The short title appears at the bottom of every slide, the full title is only on the title page

\author{María del Mar Ruiz Martín\\
        Antonio R. Moya Martín-Castaño\\
        Francisco Luque Sánchez} % Your name
\institute[UGR] % Your institution as it will appear on the bottom of every slide, may be shorthand to save space
{
  Universidad de Granada \\ % Your institution for the title page
}
\date{\today} % Date, can be changed to a custom date



\begin{document}

%% Diapositiva de título.
\begin{frame}
\titlepage 
\end{frame}

%% Diapositiva de contenidos.
% Throughout your presentation, if you choose to use \section{} and \subsection{} commands, 
% these will automatically be printed on this slide as an overview of your presentation

\section{Introducción}
\begin{frame}
  \frametitle{Introducción}
   % Table of contents slide, comment this block out to remove it
  Explicaremos el método de Euler para sistemas de ecuaciones diferenciales de primer orden. Por consiguiente, también servirá para resolver ecuaciones de orden superior.\\~\\
  
  Poco utilizado en la práctica, no aporta soluciones lo suficientemente buenas.\\~\\

  Interesante por su gran simplicidad.
  
\end{frame}


\subsection{Partes del trabajo}
\begin{frame}
	\frametitle{Partes del trabajo:}
	\begin{itemize}
		\item Definiciones previas
		\item Descripción general de un sistema de ecuaciones diferenciales
		\item Ecuaciones diferenciales de orden superior y reescritura como sistemas.
		\item Método de Euler para sistemas de ecuaciones diferenciales
		\item Estudio del error y análisis de la convergencia del método.
		\item Ejemplos
		
	\end{itemize}
\end{frame}

\section{Definiciones previas}
\begin{frame}
	\frametitle{Definiciones previas}
	
	\begin{itemize}
		\item Consistencia: Un método se dice consistente si el error local que produce tiende a 0 cuando $h$ (tamaño de paso) tienda a 0.
		$$ \lim_{h \rightarrow 0} max | \tau_i(h) | \rightarrow 0 $$
		
		\item Estabilidad: Un método se dice que es estable si una perturbación en los datos de entrada no se va maximizando cuando se utiliza el método para obtener la solución del problema. 
		
		 $$|\omega' - \omega| \leq K|X'_0 - X_0|$$.
		
		\item La consistencia y la estabilidad de un método implican su convergencia.
		
	\end{itemize}
\end{frame}

\section{Descripción general de un sistema de ecuaciones diferenciales}
\subsection{Concepto de ecuación diferencial}
\begin{frame}
	\frametitle{Concepto de ecuación diferencial:}
	
	Igualdad en la que interviene una variable independiente (\texttt{t}), una variable dependiente (x(\texttt{t})) y las sucesivas derivadas de la variable dependiente respecto de la independiente:
	
	$$
	F(t,x,x', ..., x^{n)})=0
	$$
	
	\textit{Orden} de la ecuación diferencial: mayor orden de derivación de la variable independiente que aparece en dicha ecuación.\\~\\
	
	\textit{Solución} Conjunto de ecuaciones que cumplan dichas condiciones.
\end{frame}

\subsection{PVI}
\begin{frame}
	\frametitle{Problema de valores iniciales}
	\textit{PVI} asociado a una ecuación diferencial: dicha ecuación junto con el conjunto de $n$ valores (fijado $t_0$):
	
	$$
	x(t_0)=y_0, x'(t_0)=y_1, ...,  x^{n-1)}(t_0)=y_{n-1}
	$$
	
	\textit{Problema bien planteado}: existe solución, es única y depende continuamente de los datos del problema.
	
\end{frame}

\subsection{Sistema de ecuaciones diferenciales}
\begin{frame}
	\frametitle{Sistema de ecuaciones diferenciales}
	
	Es un conjunto de ecuaciones diferenciales en las que aparecen una variable independiente, un conjunto de variables dependientes de dicha variable independiente y las sucesivas derivadas de dichas variables dependientes respecto de la independiente. 
	
	$$
	\begin{cases}
	F_1(t, x_1, x_1', ..., x_1^{n_1)}, ..., x_m, x_m', ..., x_m^{n_m)}) = 0 \\
	F_2(t, x_1, x_1', ..., x_1^{n_1)}, ..., x_m, x_m', ..., x_m^{n_m)}) = 0 \\
	\vdots \\
	F_r(t, x_1, x_1', ..., x_1^{n_1)}, ..., x_m, x_m', ..., x_m^{n_m)}) = 0
	\end{cases}
	$$
	
	PVI análogo.
\end{frame}

\begin{frame}
	\frametitle{Sistemas de primer orden}

	Las derivadas que aparecen son a lo sumo de orden 1.\\~\\
	
	$$
	\begin{cases}
	F_1(t, x_1, ..., x_m) = \frac{d x_1}{d t} \\
	F_2(t, x_1, ..., x_m) = \frac{d x_2}{d t} \\
	\vdots \\
	F_m(t, x_1, ..., x_m) = \frac{d x_m}{d t}
	\end{cases} 
	$$
	
	Y para el PVI asociado a dicho sistema: 
	
	$$ x_1(t_0) = y_1, x_2(t_0) = y_2, ..., x_m(t_0) = y_m $$
\end{frame}

\begin{frame}
    \frametitle{Sistemas de primer orden}
	El sistema anterior lo podemos notar en forma vectorial de la siguiente manera:\\~\\

	$$ X'[t] = F(t, X[t]), \quad 
	X[t] = \begin{bmatrix}
    x_1(t) \\
    \vdots \\
    x_n(t)
    \end{bmatrix}, \quad
    F[t, X[t]] = \begin{bmatrix}
    f_1(t, X[t]) \\
    \vdots \\
    f_n(t, X[t])
    \end{bmatrix} $$
    
    Está bien planteado si satisface las condiciones anteriormente dichas.
    Ahora, indicar que tenemos una caracterización de problema bien planteado a partir de la condición de Lipschitz.
\end{frame}

\begin{frame}
	\frametitle{Problema bien planteado}
	
	 Diremos que la función vectorial $F(t, X(t))$ satisface la condición de Lipschitz, con constante de Lipschitz $L$, si se cumple que $\forall t_1, t_2 \in [a,b]$ se cumple:
	 
	 $$ ||F(t_1, X(t_1)) - F(t_2, X(t_2))|| \leq L||X(t_1) - X(t_2)||$$
	 
	 Tenemos entonces que el problema anterior es un problema bien planteado si $F (t, X(t))$ es continua en su conjunto de definición y Lipschitziana.
	 
\end{frame}

\section{Ecuaciones diferenciales de orden superior y reescritura como sistemas}
\subsection{Reescritura}
\begin{frame}
	\frametitle{Reescritura de ecuaciones de orden superior}

	Sea la ecuación diferencial de orden $n$: $ F(t, x, x', ..., x^{n)}) = 0 $. Cambios de variable:

	$$y_0 = x, y_1 = x', y_2 = x'', \dots, y_{n-1} = x^{n-1)}$$\\~\\

	Solución de $y_0$ del sistema $\rightarrow$ solución de la ecuación diferencial.

\end{frame}

\begin{frame}
	
	Sistema en el que hemos transformado la ecuación de orden $n$ sería el siguiente:
	
	$$
	\begin{cases}
	y_0' = x' = y_1\\
	y_1' = x'' = y_2\\
	\vdots\\
	y_{n-2}' = x^{n-1)} = y_{n-1}\\
	y_{n-1}' = x^{n)} = F(t, y_1, y_2, \dots, y_{n-1})
	\end{cases}
	$$
	
\end{frame}

\section{Método de Euler para sistemas de ecuaciones diferenciales}
\subsection{Recordatorio previo}
\begin{frame}
	\frametitle{Método de Euler para ecuaciones diferenciales}
	
	Dado un PVI bien planteado:
	$$
	\begin{cases}
	y'(t)=f(t,y(t)) \\ 
	y(t_0)=y_0, 
	\end{cases}
	$$
	
	Dado [a,b], tomamos $N \in \mathbb{N}$, y $h=(b-a)/N$\\~\\
	Para cada $i=0,...,N$, tomamos los puntos: $t_i=a+ih$.
	
\end{frame}

\begin{frame}
	\frametitle{Método de Euler para ecuaciones diferenciales}
	Aplicando Taylor (centrado en $t_i$):

	$$y(t_{i+1})=y(t_i) + y'(t_i)(t_{i+1}-t_i) + y''(\xi_i)(t_{i+1}-t_i)^2/2$$ 

	para algún $\xi_i$ en $(t_i, ti_{i+1})$.\\~\\
	
	Equivalentemente:
	
	$$y(t_{i+1})=y(t_i) + f(t_i,y(t_i))h + y''(\xi_i)h^2/2$$
	
	Es decir: 
	
	$$y(t_{i+1})=y(t_i) + f(t_i,y(t_i))h + O(h^2)$$ 
	
\end{frame}

\begin{frame}
	\frametitle{Método de Euler para ecuaciones diferenciales}
	
	De este modo, el método aproxima cada $y_i$:
	
	$$
	\begin{cases}
	w_0=y_0\\
	w_{i+1}=w_i + hf(t_i,w_i)
	\end{cases}
	$$
\end{frame}

\subsection{Método de Euler para sistemas de ecuaciones diferenciales}
\begin{frame}
	\frametitle{Método de Euler para sistemas de ecuaciones diferenciales}
	
	Dicho esto, podemos observar, mediante una deducción análoga, cuál es el punto de partida del método de Euler para sistemas. Dado el sistema:
	
	$$
	\begin{cases}
	x_1'=F_1(t,x_1(t),...,x_n(t)) \\
	\vdots\\
	x_n'=F_n(t,x_1(t),...,x_n(t))
	\end{cases}
	$$
	
 para cada $j=1,...,n$, podemos hacer una aproximación de $w_{i,j} \approx x_j(t_i)$ de forma que:
 $$
 \begin{cases}
 w_{0,j}=x_{j,0} \\
 w_{i+1,j}=w_{i,j}+ hF_j(t_i,w_{i,j})
 \end{cases}
 $$ 
\end{frame}

\begin{frame}
	\frametitle{Método de Euler para sistemas de ecuaciones diferenciales}
	
	Y así, dado el siguiente sistema: $$
	X'(t)=F(t,X(t))
	$$
	donde X(t) es de la forma:
	
	\begin{equation*}
	X(t)=\begin{bmatrix}
	x_1(t) \\
	\vdots \\
	x_n(t)
	\end{bmatrix}
	\end{equation*}
	y F(t,X(t)):
	\begin{equation*}
	F(t,X(t))=\begin{bmatrix}
	F_1(t,x_1(t),...,x_n(t)) \\
	\vdots \\
	F_n(t,x_n(t),...,x_n(t))
	\end{bmatrix}
	\end{equation*}
	
	
	 
\end{frame}

\begin{frame}
	\frametitle{Método de Euler para sistemas de ecuaciones diferenciales}
	
	podemos expresar una aproximación $W_i \approx X(t_i)$ de a siguiente manera:
	$$
	\begin{cases}
	W_0=X(t_0)\\
	W_{i+1}=W_i + hF(t_i,W_i)
	\end{cases}
	$$
	donde $W_i$ es de la forma:
	
	\begin{equation*}
	W_i=\begin{bmatrix}
	w_1(t_i) \\
	\vdots \\
	w_n(t_i)
	\end{bmatrix}
	\end{equation*}
	
	
\end{frame}

\section{Estudio del error y análisis de la convergencia del método}

\subsection{Tipos de errores}
\begin{frame}
	\frametitle{Tipos de errores en los métodos de resolución numérica}

	\begin{itemize}
	\item Error local
	\item Error global
	\item Error de truncatura
	\end{itemize}

\end{frame}

\subsection{Error local}
\begin{frame}
	\frametitle{Error local}
	\begin{block}{Teorema}
	El error local para el método de Euler es $O(h^2)$, donde $h$ es el tamaño del paso que hemos elegido para aplicar el método.
	\end{block}
	
	Para la demostración, supongamos que tenemos un valor exacto para un determinado $t_i$, entonces el error local para la siguiente iteración es $|| E_{i+1} || = ||X(t_{i+1}) - [X(t_{i}) + h*F(t_i, X_i)] ||$
	
	Si tomamos cada componente por separado y, además, realizamos el desarrollo de Taylor de $x_j(t_{i+1})$ con $j=1,...,n$, llegaríamos a que para cada $x_j$, el error cometido en un paso del método es:
	$$ e_{i+1, j} = | x_j(t_{i+1}) - \omega_{i+1, j} | = \frac{h^2}{2}*x''_j(\xi_{i+1, j}),\, \xi_{i+1, j} \in [t_i, t_{i+1}] $$	
	
\end{frame}

\begin{frame}
	\frametitle{Error local}
	
	Tenemos entonces que el vector de los errores lo podemos expresar como sigue:
	
	\begin{equation*}
	E_{i+1}=\begin{bmatrix}
	e_{i+1, 1} \\
	\vdots \\
	e_{i+1, n}
	\end{bmatrix}=\begin{bmatrix}
	\frac{h^2}{2}*x''_1(\xi_{i+1, 1}) \\
	\vdots \\
	\frac{h^2}{2}*x''_n(\xi_{i+1, n})
	\end{bmatrix}
	\end{equation*}
	
	Tomando ahora la norma euclídea del vector de los errores, tenemos que el error es $O(h^2)$
\end{frame}

\subsection{Error global}
\begin{frame}
	\frametitle{Error global}
	
	\begin{block}{Lema 1}
		Para toda $x \geq -1$ y para cualquier $m$ positiva, tenemos que $0 \leq (1 + x)^m \leq e^{mx}$
	\end{block}
	
	
	\begin{block}{Lema 2}
		Si $s$ y $t$ son números reales positivos, $\{a_i\}^k_{i=0}$ es una sucesión que satisface $ a_0 \geq -t/s$ y
		$$a_{i+1} \leq (1+s)a_i + t, \, \forall i=0, 1, ..., k$$
		entonces se tiene que
		$$a_{i+1} \leq e^{(i+1)s}\left( a_0 + \frac{t}{s}\right) - \frac{t}{s}$$
	\end{block}
	%Aplicando el teorema de Taylor:
	
	%$$ 0 \leq 1 + x \leq 1 + x + \frac{1}{2}x^2e^\xi = e^x $$
	
	%donde $0 < \xi < x$.
	
	%y como $1 + x \geq 0$:
	%$$ 0 \leq (1 + x)^m \leq (e^x)^m = e^{mx} $$
\end{frame}

\begin{frame}
	\frametitle{Error global}
	\begin{block}{Teorema}
		El error global del método de Euler para sistemas de ecuaciones es $O(h)$, donde $h$ es el tamaño de paso que hemos elegido al aplicar el método. 
	\end{block}
	
	El esquema general de esta demostración sería el siguiente. Suponiendo que cada función, $f_j(t, x_1, ..., x_n)$ es Lipschitziana en la variable $x_j$, con constante de Lipschitz $L_j$, y que para cada $x_j$ existe una constante $M_j$ tal que $| x''_j(t) | \leq M_j, \forall t \in [a,b]$.\\
	
	
\end{frame}

\begin{frame}
	\frametitle{Error global}

	Teniendo en cuenta el desarrollo de Taylor para $x_j(t_{i+1})$ y la aproximación $w_{i+1,j}$:
	
	$$
	x_j(t_{i+1}) - \omega_{i+1,j} = x_j(t_i) - \omega_{i,j} + h*[f_j(t_i, x_i, ..., x_n) $$
	$$- f_j(t_i, \omega_{i,1}, ..., \omega_{i,n})] + \frac{h^2}{2}*x''_j(\xi_j) $$
	Se cumple entonces que:
	$$ | x_j(t_{i+1}) - \omega_{i+1,j} | \leq | x_j(t_i) - \omega_{i,j} | + h|[f_j(t_i, x_i, ..., x_n) $$
	$$- f_j(t_i, \omega_{i,1}, ..., \omega_{i,n})]| + \frac{h^2}{2}*|x''_j(\xi_j)| $$
	
\end{frame}

\begin{frame}
	\frametitle{Error global}
	Y de aquí, teniendo en cuenta que $f_j$ es Lipschitziana:
	
	$$ | x_j(t_{i+1}) - \omega_{i+1,j} | \leq (1+hL_j)*| x_j(t_i) - \omega_{i,j} | + \frac{h^2*M_j}{2} $$
	
	Y aplicando el lema anterior y operando:
	
	$$ | x_j(t_{i+1}) - \omega_{i+1, j} | \leq \frac{hM_j}{2L_j}(e^{(t_{i+1}-a)L_j} - 1) $$
	Y dado que tomamos $j$ arbitrario, tenemos este resultado para todas las funciones $f_j$. Finalmente, tomando la norma euclídea de todos los errores globales, podemos concluir que el método es de orden $O(h)$.
\end{frame}

\subsection{Error de truncatura}
\begin{frame}
	\frametitle{Error de truncatura}

	Errores derivados de la precisión con la que trabaja la máquina en la que calculamos.\\~\\

	Este error provoca que se aumente el error total para valores de $h$ demasiado pequeños.

\end{frame}

\begin{frame}
	\frametitle{Ejemplo del error de truncatura}

		Se ha cambiado la precisión de la máquina para que trabaje con 6 decimales significativos.
		\begin{figure}
		\centering
		\includegraphics[scale=0.4]{img/graph_error_comb}
		\end{figure}

\end{frame}

\section{Ejercicios}

\subsection{Método de Euler para sistemas de ecuaciones de primer orden}

\begin{frame}
	\frametitle{Método de Euler para sistemas de ecuaciones primer orden}
	\begin{block}{Ejercicio 1}
		Dado el siguiente PVI, determinar el valor de x(1) e y(1):
		
		$$
		\begin{cases}
		x' = 2*x + 3 * y\\
		y' = 2*x + y \\
		x(0) = 4\\
		y(0)=1
		\end{cases}
		$$
	\end{block}
	
	Veamos el cálculo de una primera aproximación $X_1$
	
	\begin{equation*}
	X_1=X_0+f(t_0,X_0)h=\begin{bmatrix}
	4\\
	1\\
	\end{bmatrix} + \begin{bmatrix}
	2*4 +3*1\\
	2*4 + 1
	\end{bmatrix}0.1=\begin{bmatrix}
	5.1\\
	1.9
	\end{bmatrix}  
	\end{equation*}
	
\end{frame}

\begin{frame}
	\frametitle{Método de Euler para sistemas de ecuaciones de primer orden}
	
	Observemos ahora:
	
	\begin{table}[H]
		\centering
		\setlength\extrarowheight{2.5pt}
		
		\begin{tabular}{|c|c|c|c|c|c}
			\hline
			\textbf{h} & {\textbf{$w_1(1)$}} & \textbf{Err en $w_1(1)$} & {\textbf{$w_2(1)$}} & \textbf{Err en $w_2(1)$} \\ 
			\hline
			0.1 & 121.80076 & 42.36156 & 80.67749 & 28.15092\\
			\hline
			0.01 & 151.88087 & 12.28145 & 100.64386 & 8.18455\\
			\hline
			0.001 & 162.86047 & 1.30185 & 107.96082 & 0.86759\\
			\hline
			0.0001 & 164.09679 & 0.06553 & 108.78482 & 0.04360\\
			\hline
			0.00001 & 164.15577 & 0.00655 & 108.82405 & 0.00436\\
			\hline   
		\end{tabular}
		
		\caption{Errores obtenidos con el método de Euler}           
	\end{table}
	
\end{frame}

\begin{frame}
	\frametitle{Método de Euler para sistemas de ecuaciones de primer orden}
	
	\begin{figure}[H]
		\centering
		\includegraphics[scale=0.4]{img/grafica3.png}
		\label{figura1}
		\caption{Aproximación para marcas de paso de 0.1, 0.05 y 0.01 respectivamente} 
	\end{figure}
	
\end{frame}

\subsection{Método de Euler para ecuaciones de orden superior}

\begin{frame}
	\frametitle{Método de Euler para ecuaciones de orden superior}


\begin{block}{Ejercicio 2}
	Dado el siguiente PVI, determinar el valor de x($\pi/4$)
	$$
	x''' = 2x'' - 2x'     % x(t) = e^tcos(t)
	$$
	$$
	x(0)=1, x'(0)=1, x''(0)=0
	$$
\end{block}

$$
\begin{cases}
y_0' = x' = y_1\\
y_1' = x'' = y_2\\
y_2' = x''' = 2y_2 - 2y_1
\end{cases}
$$

\end{frame}

\begin{frame}
	\frametitle{Método de Euler para ecuaciones de orden superior}

	\begin{figure}[H]
	\centering
	\includegraphics[scale=0.45]{img/grafica1.png}
	\label{figura1}
	\caption{Aproximación para 25 (w1), 50 (w2) y 100 (w3) nodos} 
	\end{figure}


\end{frame}


\begin{frame}
	\frametitle{Método de Euler para ecuaciones de orden superior}

    \begin{table}[H]
        \centering
        \setlength\extrarowheight{3pt}
        
        \begin{tabular}{|c|c|c|c|c}
            \hline
            \textbf{h} & {\textbf{w($\pi$/4)}} & \textbf{Error} \\ 
            \hline
                $\pi$/120 & 1.581351427438043 & 0.030468230520017\\
            \hline
                $\pi$/240 & 1.566883575533628 & 0.016000378615602\\
            \hline
                $\pi$/360 & 1.561353411976403 & 0.010470215058378\\
            \hline
                $\pi$/480 & 1.558765635465628 & 0.007882438547603\\
            \hline
                $\pi$/600 & 1.557203467884946 & 0.0063202709669205\\
            \hline
                $\pi$/720 &  1.556204299202344 & 0.0053211022843184\\
            \hline
                $\pi$/840 & 1.555409383683734 & 0.0045261867657087\\
            \hline
        \end{tabular}
        
        \caption{Errores obtenidos con el método de Euler}           
    \end{table}

\end{frame}

\begin{frame}
\Huge{\centerline{Fin.}}
\end{frame}

%----------------------------------------------------------------------------------------

\end{document} 